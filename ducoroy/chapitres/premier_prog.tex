\chapter{Votre premier \prog}
\section{Le monde merveilleux de la console}
Quand je vous annonce que nous allons commencer à \progrer, vous vous dites sûrement\textit{ « Chouette, je vais pouvoir faire ça, ça et ça ; et j'ai toujours rêvé de faire ça aussi ! »}. Il est de mon devoir de calmer un peu vos ardeurs et de vous expliquer comment cela va se passer.

Nous allons commencer doucement. Nous n'avons de toute façon pas le choix car les \progs complexes 3D en réseau que vous imaginez peut-être nécessitent de connaître les bases.

Il faut savoir qu'il existe 2 types de \progs : les \progs graphiques et les \progs console.
\subsection{Les \progs graphiques}
Il s'agit des \progs qui affichent des fenêtres. Ce sont ceux que vous connaissez sûrement le mieux. Ils génèrent à l'écran des fenêtres que l'on peut ouvrir, réduire, fermer, agrandir…
Les programmeurs parlent de \emph{GUI} (Graphical User Interface - Interface Utilisateur Graphique voir \fig \ref{gui}). 
\fimg{ducoroy/images/gui}{Un programme \textsc{gui} (graphique)}{gui}

\subsection{Les \progs console}
Les \progs en console sont plus fréquents sous \textbf{Linux} que sous \textbf{Windows} et \textbf{Mac OS X}. Ils sont constitués de simples textes qui s'affichent à l'écran, le plus souvent en blanc sur fond noir (\fig \ref{console}).

\fimg{ducoroy/images/console}{Un programme en console)}{console}
Ces \progs fonctionnent au clavier. La souris n'est pas utilisée.
Ils s'exécutent généralement linéairement : les messages s'affichent au fur et à mesure, de haut en bas.
\subsection{Notre première cible : les \progs console}
Je vous annonce que nous allons commencer par réaliser des \progs console. En effet, bien qu'ils soient un peu austères \textit{à priori}, ces \progs sont beaucoup plus simples à créer que les \progs graphiques.
\section{Explications sur ce premier code source}
Lorsque \textbf{Code::Blocks} crée un nouveau projet, il génère un fichier \lstinline|main.cpp| contenant ce code :

\listinfo{C++}{listings/codes/premier_code.cpp}






\subsection{include}
La toute première ligne est : \partlist{C++}{1-1}{./listings/codes/premier_code.cpp}
C'est ce qu'on appelle une \emph{directive de préprocesseur}. Son rôle est de \guill{charger} des fonctionnalités du \cplus pour que nous puissions effectuer certaines actions.

En effet, \emph{le \cplus est un \lang très modulaire}. De base, il ne sait pas faire grand-chose (pas même afficher un message à l'écran !). On doit donc charger des extensions que l'on appelle bibliothèques et qui nous offrent de nouvelles possibilités.

Ici on charge le fichier \lstinline!iostream!, ce qui nous permet d'utiliser une bibliothèque… d'affichage de messages à l'écran dans une console !

\subsection{using namespace}
\partlist{C++}{2-2}{./listings/codes/premier_code.cpp}
… permet en quelque sorte d'indiquer dans quel lot de fonctionnalités notre fichier source va aller piocher.

Si vous chargez plusieurs bibliothèques, chacune va proposer de nombreuses fonctionnalités. Parfois, certaines fonctionnalités ont le même nom. Imaginez une commande \guill{AfficherMessage} qui s'appellerait ainsi pour \lstinline|iostream| mais aussi pour \lstinline|Qt| ! Si vous chargez les deux bibliothèques en même temps et que vous appelez \guill{AfficherMessage}, l'\ordi ne saura pas s'il doit afficher un message en console avec \lstinline|iostream| ou dans une fenêtre avec \lstinline|Qt| !

Pour éviter ce genre de problèmes, on a créé des \emph{namespaces} (espaces de noms), qui sont des sortes de dossiers à noms. La ligne \lstinline|using namespace std;| indique que vous allez utiliser l'espace de noms \lstinline|std| dans la suite de votre fichier de code. Cet espace de noms est un des plus connus car il correspond à la bibliothèque standard (\lstinline|std|), une bibliothèque livrée par défaut avec le \lang \cplus et dont \lstinline|iostream| fait partie.

\subsection{int main}
C'est ici que commence vraiment le cœur du \prog. Les \progs, vous le verrez, sont essentiellement constitués de fonctions. Chaque fonction a un rôle et peut appeler d'autres fonctions pour effectuer certaines actions.
Tous les \progs possèdent une fonction dénommée \guill{main} (Qui se prononce \guill{mèïne} en anglais.), ce qui signifie \guill{principale}. C'est donc la fonction principale.

Une fonction à la forme suivante :

\listinfo{C++}{listings/codes/codevide.cpp}
Au bout de la fonction \lstinline|main| le \prog s'arrête ! Tout \prog commence au début de la fonction \lstinline|main| et se termine à la fin de celle-ci.

\subsection{cout}
Voici enfin la première ligne qui fait quelque chose de concret ! C'est la première ligne de \lstinline|main|, donc la première action qui sera exécutée par l'\ordi (les lignes que nous avons vues précédemment ne sont en fait que des préparatifs pour le \prog).

\partlist{C++}{5-5}{./listings/codes/premier_code.cpp}
Le rôle de \lstinline|cout| (à prononcer \guill{ci aoute}) est d'afficher un message à l'écran. C'est ce qu'on appelle une instruction. Tous nos \progs seront constitués d'instructions comme celle-ci, qui donnent des ordres à l'\ordi.

Notez que \lstinline|cout| est fourni par \lstinline|iostream|. Si vous n'incluez pas \lstinline|iostream| au début de votre \prog, le compilateur se plaindra de ne pas connaître \lstinline|cout| et vous ne pourrez pas générer votre \prog !

\subsection{return}
La dernière ligne est :
\partlist{C++}{6-6}{./listings/codes/premier_code.cpp}

Ce type d'instruction clôt généralement les fonctions. En fait, la plupart des fonctions renvoient une valeur (un nombre par exemple). Ici, la fonction \lstinline|main| renvoie \nb{0} pour indiquer que tout s'est bien passé (toute valeur différente de \nb{0} aurait indiqué un problème).

Vous n'avez pas besoin de modifier cette ligne, laissez-la telle quelle. Nous aurons d'autres occasions d'utiliser \lstinline|return| pour d'autres fonctions, nous en reparlerons !

\section{Commentez et mettez en forme vos \progs !}
En plus du code qui donne des instructions à l'\ordi, vous pouvez écrire des commentaires pour expliquer le fonctionnement de votre \prog.

Les commentaires n'ont aucun impact sur le fonctionnement de votre logiciel : en fait, le compilateur ne les lit même pas et ils n'apparaissent pas dans le \prog généré. Pourtant, ces commentaires sont indispensables pour les développeurs : ils leur permettent d'expliquer ce qu'ils font dans leur code !

\subsection{Différents types de commentaires}
Il y a deux façons d'écrire des commentaires selon leur longueur. Je vais vous les présenter toutes les deux.

\subsubsection{Commentaires courts}
Pour écrire un commentaire court, sur une seule ligne, il suffit de commencer par \lstinline|//| puis d'écrire votre commentaire. Cela donne :

\partlist{C++}{1-1}{./listings/codes/commentaires.cpp}

Mieux, vous pouvez aussi ajouter le commentaire à la fin d'une ligne de code pour expliquer ce qu'elle fait :

\partlist{C++}{2-2}{./listings/codes/commentaires.cpp}

\subsubsection{Commentaires longs}

\partlist{C++}{3-6}{./listings/codes/commentaires.cpp}

