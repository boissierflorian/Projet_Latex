\chapter{Un peu de mathématiques}
\section{L'en-tête \lstinline|cmath|}
Pour avoir accès à plus de fonctions mathématiques, il suffit d'ajouter une ligne en tête de votre \prog, comme lorsque l'on désire utiliser des \varis de type \lstinline|string|. La directive à insérer est :

\partlist{C++}{2-2}{ducoroy/listings/codes/maths.cpp}

Commençons avec une fonction utilisée très souvent, la racine carrée. En anglais, racine carrée se dit \textit{square root} et, en abrégé, on écrit parfois \lstinline|sqrt|. Comme le \cplus utilise l'anglais, c'est ce mot là qu'il va falloir retenir et utiliser.

Pour utiliser une fonction mathématique, on écrit le nom de la fonction suivi, entre parenthèses, de la valeur à calculer. On utilise alors l'affectation pour stocker le résultat dans une \vari.

C'est comme en math quand on écrit $y = f(x)$. Il faut juste se souvenir du nom compliqué des fonctions. Pour la racine carrée, cela donnerait : \partlist{C++}{8-8}{ducoroy/listings/codes/maths.cpp}

Prenons un exemple complet, je pense que vous allez comprendre rapidement le principe.

\listinfo{C++}{listings/codes/maths.cpp}

\subsection{Le cas de la fonction puissance}
Comme toujours, il y a un cas particulier : la fonction puissance. Comment calculer $4^5?$

Si je veux calculer $4^5$, je vais devoir procéder ainsi :

\listinfo{C++}{ducoroy/listings/codes/calcul_puissance.cpp}