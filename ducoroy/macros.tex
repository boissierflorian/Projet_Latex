%Raccourcis de langage
\newcommand{\cplus}{\textbf{\emph{C++}}\xspace}
\newcommand{\lang}{langage\xspace}
\newcommand{\langs}{langages\xspace}
\newcommand{\nb}[1]{\num{#1}\xspace}
\newcommand{\progio}{programmation\xspace}
\newcommand{\prog}{programme\xspace}
\newcommand{\progs}{programmes\xspace}
\newcommand{\progrer}{programmer\xspace}
\newcommand{\ordi}{ordinateur\xspace}
\newcommand{\ordis}{ordinateurs\xspace}
\newcommand{\fig}{figure\xspace}
\newcommand{\vari}{variable\xspace}
\newcommand{\varis}{variables\xspace}

%Mettre un élément, passé en argument, entre guillemets
\newcommand{\guill}[1]{\og #1 \fg{}\xspace}


% Listing créé à partir du #1 : langage souhaité, #2 : fichier ciblé
\newcommand{\listinfo}[2]{
	\lstset{language=#1,
		numbers=left,
		frame=single,
		framesep=2pt,
		aboveskip=1ex,
		basicstyle=\ttfamily,
		keywordstyle=\color{blue},
		stringstyle=\color{red},
		commentstyle=\color{green},
		inputencoding=utf8/latin1,
		breaklines=true
	}
	\lstinputlisting{#2}
}	

% Génère un fragment de code #1 : langage souhaité,
%#2 : Lignes à extraire du code, sous format : num ligne - num ligne, ex : 1-1
%#3 : fichier ciblé

\newcommand{\partlist}[3]{
	\lstset{language=#1,
		basicstyle=\ttfamily,
		frame=single,
		framesep=2pt,
		aboveskip=1ex,
		keywordstyle=\color{blue},
		stringstyle=\color{red},
		inputencoding=utf8/latin1,
		commentstyle=\color{green}
	}
		\lstinputlisting[linerange={#2}]{#3}
}

%Insertion d'images flottantes centrées de 7cm
\newcommand{\fimg}[3]{
	\begin{figure}[ht]
		\centering
		\includegraphics[height=5cm,width=0.750\linewidth]{#1}
		\caption{#2}
		\label{#3}
	\end{figure}
}

