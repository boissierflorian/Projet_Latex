%%%%%%%%%%%%%%%%%%%%%%%%%%%%%%%%%%%%%%%%%%%%%%%%
% Boissier Florian
% Fichier personel principal: boissier/rapport.tex
% L3 Info
%%%%%%%%%%%%%%%%%%%%%%%%%%%%%%%%%%%%%%%%%%%%%%%%

\chapter{Quaternions et rotation dans l'espace}

Les quaternions unitaires fournissent une notation mathématique commode pour représenter l'orientation et la rotation d'objets en trois dimensions. Comparés aux angles d'Euler, ils sont plus simple à composer et évitent le problème du blocage de cardan. Comparés aux matrices de rotations, ils sont plus stables numériquement et peuvent se révéler plus efficaces. Les quaternions ont été adoptés dans des applications en infographie, robotique, navigation, dynamique moléculaire et la mécanique spatiale des satellites.

\section{Opérations de rotation à l'aide de quaternions}
	\subsection{L'hypersphère des rotations}
		\subsubsection{Se faire une idée de l'espace des rotations}
		\subsubsection{Paramétrer l'espace des rotations}
	\subsection{Des rotations aux quaternions}
		\subsubsection{Les quaternions en bref}
		\subsubsection{Relation entre les rotations et les quaternions unitaires}
		\subsubsection{Démonstration de l'équivalence entre conjugaison de quaternions et rotation de l'espace}
	\subsection{Exemple}
	
\section{Expliquer les propriétés des quaternions à l'aide des rotations}
	\subsection{Non-commutativité}
	\subsection{Les quaternions sont-ils orientés ?}
	
\section{Les quaternions et les autres représentations des rotations}
	\subsection{Description qualitative des avantages des quaternions}
	\subsection{Conversion vers et depuis la représentation sous forme de matrice}
		\subsubsection{D'un quaternion en matrice orthogonale}
		\subsubsection{D'une matrice orthogonale en quaternion}
		\subsubsection{Quaternions optimaux}
	\subsection{Comparaisons de performances avec d'autres méthodes de rotation}
		\subsubsection{Résultats}
		\subsubsection{Méthodes utilisées}
\section{Les paires de quaternions unitaires comme rotations dans l'espace à 4 dimensions}