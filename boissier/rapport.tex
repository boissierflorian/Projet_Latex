%%%%%%%%%%%%%%%%%%%%%%%%%%%%%%%%%%%%%%%%%%%%%%%%
% Boissier Florian
% Fichier personel principal: boissier/rapport.tex
% L3 Info
%%%%%%%%%%%%%%%%%%%%%%%%%%%%%%%%%%%%%%%%%%%%%%%%

\chapter{Quaternions et rotation dans l'espace}
Les quaternions unitaires fournissent une notation mathématique commode pour 
représenter l'orientation et la rotation d'objets en trois dimensions. 
Comparés aux angles d'Euler, ils sont plus simple à composer et évitent le problème 
du blocage de cardan. Comparés aux matrices de rotations, ils sont plus stables numériquement et peuvent se révéler plus efficaces. Les quaternions ont été adoptés 
dans des applications en infographie, robotique, navigation, dynamique moléculaire 
et la mécanique spatiale des satellites.

\section{Opérations de rotation à l'aide de quaternions}
	Une explication très stricte des propriétés utilisées dans cette section est 
donnée par \href{http://www.simonaltmann.com/Simon_Altmann/Welcome.html}{Altmann}
	\subsection{L'hypersphère des rotations}
		\subsubsection{Se faire une idée de l'espace des rotations}
			Les quaternions unitaires représentent l'\emph{espace mathématique} des rotations en trois dimensions de façon relativement simple. On peut comprendre la correspondance entre les rotations et les quaternions en commençant d'abord par se faire une idée intuitive de l'espace des rotations lui-même.
Deux rotations d'angles différents et d'axes différents dans l'espace des rotations. La norme du vecteur est liée à l'amplitude de la rotation.

Chaque rotation en trois dimensions consiste à tourner d'un certain angle autour d'un certain axe. Quand l'angle est nul, l'axe n'a pas d'importance, de telle façon qu'une rotation de zéro degré est un simple point dans l'espace des rotations (c'est la rotation identité). Pour un angle petit mais non nul, l'ensemble des rotations possibles est une petite sphère entourant la rotation identité, où chaque point de la sphère représente un axe pointant dans une direction particulière (comparez avec la sphère céleste). Des rotations d'angles de plus en plus grands s'éloignent progressivement de la rotation identité, et nous pouvons nous les représenter comme des sphères concentriques de rayons croissants. Par conséquent, au voisinage de la rotation identité, l'espace abstrait des rotations ressemble à l'espace ordinaire en trois dimensions (qui peut également être vu comme un point central entouré de sphères de différents rayons. La ressemblance s'arrête là : lorsque l'angle de rotation dépasse \ang{180}, les rotations suivant les différents axes cessent de diverger et commencent à nouveau à se ressembler, pour finir par devenir identiques (et égales à la rotation identité) lorsque l'angle atteint \ang{360}.

\begin{figure}[ht]
	\centering
	\includegraphics[width=5cm]{ressources/espace_rotations}\hfill
	\caption{Deux rotations d'angles différents et d'axes différents dans l'espace des rotations. La norme du vecteur est liée à l'amplitude de la rotation.}
	\label{espace_rotations}
\end{figure}

L'hypersphère des rotations pour les rotations d'axes horizontaux (axes compris dans le plan xy).
On constate un phénomène analogue à la surface d'une sphère. Si nous nous plaçons au pôle Nord et traçons à partir de là des lignes droites (en fait, des méridiens) dans plusieurs directions, elles divergeront puis convergeront à nouveau au pôle Sud. Des cercles concentriques de rayon croissant dessinés autour du pôle Nord (des parallèles) finiront par s'effondrer en un point au pôle Sud une fois que l'on a parcouru la distance entre les pôles. On peut assimiler les différentes directions à partir du pôle (c'est-à-dire les différents méridiens) aux différents axes de rotations et les différentes distances au pôle Nord aux différents angles : on a ainsi une analogie de l'espace des rotations. Mais la surface de la sphère est en deux dimensions alors que les \emph{axes} de rotation utilisent déjà trois dimensions. L'espace des rotations est donc modélisé par une sphère de dimension 3 dans un espace à 4 dimensions (une hypersphère). Nous pouvons penser à la sphère ordinaire comme à une section de l'hypersphère, de la même façon qu'un cercle est une section de sphère. On peut prendre la section pour représenter, par exemple, uniquement les rotations d'axes dans le plan $xy$ (voir illustration ci-contre). On remarque que l'angle de la rotation est deux fois la différence de latitude avec le pôle Nord : en effet, les points de l'équateur représentent des rotations de \ang{180}, pas de \ang{90}, et le pôle Sud représente la rotation identité de \ang{360}, et pas le demi-tour de \ang{180}.

Le pôle Nord et le pôle Sud représentent la même rotation, et en fait cela s'applique à n'importe quelle paire de points aux antipodes l'un de l'autre : si un point correspond à une rotation d'angle $\alpha$ autour de l'axe dirigé par le vecteur $\vv{v}$, l'autre point correspond à une rotation d'angle $\text{\ang{360}} - \alpha$ autour de l'axe dirigé par le vecteur $\vv{v}$. En fait, l'espace des rotations n'est pas l'hypersphère elle-même, mais l'hypersphère où l'on identifie les points aux antipodes l'un de l'autre. Mais dans un but de simplification, nous pouvons penser aux rotations comme à des points de la sphère en dimension 4, même si la moitié de ces points est redondante (revêtement double).

\begin{figure}[ht]
	\centering
	\includegraphics[width=5cm,height=4cm]{ressources/hypersphere_rotations}\hfill
	\caption{L'hypersphère des rotations pour les rotations d'axes horizontaux (axes compris dans le plan $xy$).}
	\label{hypersphere_rotations}
\end{figure}
		\subsubsection{Paramétrer l'espace des rotations}
			Nous pouvons paramétrer la surface d'une sphère à l'aide de deux coordonnées, comme la latitude et la longitude. Mais la latitude et la longitude se comportent mal (sont dégénérés) aux pôles Nord et Sud, alors que les pôles ne sont pas différents par nature des autres points de la sphère. Aux pôles Nord et Sud (de latitudes $+$\ang{90} et $-$\ang{90}), la longitude perd son sens.

On peut montrer qu'aucun système de coordonnées à deux paramètres ne peut éviter cette dégénérescence (c'est le théorème de la boule chevelue). Nous pouvons éviter de tels problèmes en plongeant la sphère dans l'espace à trois dimensions et en la paramétrant au moyen de trois coordonnées cartésiennes (ici $w$, $x$ et $y$), en plaçant le pôle Nord à $(w, x, y) = (1, 0, 0)$, le pôle Sud à $(w, x, y) = ( -1, 0, 0)$ et l'équateur sera le cercle d'équations $w = 0 \text{et} x^2 + y^2 = 1$. Les points de la sphère satisfont la contrainte $w^2 + x^2 + y^2 = 1$, donc nous avons toujours deux degrés de liberté, bien que l'on ait trois coordonnées. Un point $(w, x, y)$ de la sphère représente une rotation de l'espace ordinaire autour de l'axe horizontal dirigé par le vecteur $\vv{v}
\left(
\begin{tabular}{c}
	x \\ y \\ 0 \\
\end{tabular}
\right)$
et d'angle $\alpha = 2\cos^{-1} w = 2 \sin^{-1}\sqrt{x^2+y^2}$.

De la même façon, l'hypersphère décrivant l'espace des rotations dans l'espace en trois dimensions peut être paramétrée au moyen de trois angles (angles d'Euler), mais tout paramétrage de ce type dégénère en certains points de l'hypersphère, ce qui conduit au problème du blocage de cardan. Nous pouvons éviter cela en utilisant quatre coordonnées euclidennes $w, x, y et z$, avec $w^2 + x^2 + y^2 + z^2 = 1$. Le point de coordonnées $(w, x, y, z)$ représente une rotation autour de l'axe dirigé par le vecteur $\vv{v}
\left(
\begin{tabular}{c}
x \\ y \\ z \\
\end{tabular}
\right)$
et d'angle $\alpha = 2\cos^{-1} w = 2 \sin^{-1}\sqrt{x^2+y^2+z^2}$.
	\subsection{Des rotations aux quaternions}
		\subsubsection{Les quaternions en bref}
			On peut définir les nombres complexes en introduisant un symbole abstrait $i$ qui se conforme aux 
règles usuelles de l'algèbre et qui en plus obéit à la règle $i^2 = - 1$. 
Cela suffit à reproduire toutes les règles de calcul des nombres complexes, 
par exemple: 
\[
(a + bi)(c + di) = ac + adi + bic + bidi = ac + adi + bci + bdi^{2} = (ac - bd) + (bc + ad) i
\]

De la même façon, les quaternions peuvent être définis en introduisant des symboles abstraits $i, j \text{et} k$ qui satisfont aux règles $i^2 = j^2 = k^2 = ijk = -1$ et les règles algébriques usuelles \emph{sauf} la commutativité de la multiplication (un exemple familier de multiplication non commutative est la multiplication des matrices). L'ensemble des règles de calcul découle de ces définitions ; par exemple, on peut montrer que 
\[
(a+bi+cj+dk)(e+fi+gj+hk)=(ae-bf-cg-dh)+(af+be+ch-dg)i+(ag+ce+df-bh)j+(ah+de+bg-cf)k.
\]

La partie imaginaire $bi+cj+dk$ d'un quaternion se comporte comme un vecteur \vvv{v}{b}{c}{d} d'un espace vectoriel à trois dimensions et la partie réelle $a$ comme un scalaire de $\mathbb{R}$.

Quand les quaternions sont utilisés en géométrie, il est pratique de les définir comme un scalaire plus un vecteur: $a+bi+cj+dk = a + \vec{v}$

Ceux qui ont étudié les vecteurs à un niveau élémentaire pourraient trouver étrange d'additionner un \emph{nombre} à un vecteur, car ce sont des objets de natures très différentes, ou de \emph{multiplier} deux vecteurs entre eux, car cette opération n'est d'habitude pas définie. Néanmoins, si l'on se souvient qu'il ne s'agit là que d'une notation pour les parties réelles et imaginaires d'un quaternion, cela devient plus légitime.

Nous pouvons exprimer la multiplication de quaternions (produit de Hamilton) dans le langage moderne du produit vectoriel et du produit scalaire de vecteurs (qui ont en fait été inspirés au début par les quaternions). À la place des règles $i^2=j^2=k^2=ijk=-1$, nous avons la règle de multiplication de deux vecteurs $\vec{v}\vec{w}=\vec{v}\wedge\vec{w}-\vec{v}\cdot\vec{w}$, où:

\begin{itemize}
	\item $\vec{v}\vec{w}$ est la multiplication de vecteurs,
	\item $\vec{v}\wedge\vec{w}$ est le produit vectoriel (un vecteur),
	\item $\vec{v}\cdot\vec{w}$ est le produit scalaire un nombre).s
\end{itemize}

La multiplication de vecteurs n'est pas commutative (à cause du produit vectoriel), alors que la multiplication entre scalaires et entre un scalaire et un vecteur sont commutatives. Il découle de manière immédiate de ces règles que $(s + \vec{v})(t + \vec{w}) = (st - \vec{v}\cdot\vec{w}) + (s\vec{w}+t\vec{v}+\vec{v}\wedge\vec{w})$.

L'inverse (à gauche et à droite) d'un quaternion non nul est $(s + \vec{v})^-1 = 
\frac{s - \vec{v}}{s^2 + \abs{\vec{v}}^2}$, comme cela peut être vérifié par calcul direct.
		\subsubsection{Relation entre les rotations et les quaternions unitaires}
			Soient $(w, x, y, z)$ les coordonnées d'une rotation, comme décrit précédemment. 
Définissons le quaternion : $q = w + xi + yj+ zk = w + \vvv{v}{x}{y}{z} = 
\cos( \alpha / 2 ) + \vec{u} \sin( \alpha/2 )$
où $\vec{u}$ est un vecteur unitaire. Soit également $\vec{v}$ un vecteur
ordinaire de l'espace en 3 dimensions, considéré comme un quaternion avec
une coordonnée réelle nulle. On peut alors montrer (voir section suivante) 
que le produit de quaternions: 
\[
q\vec{v}q^-1
\]
renvoie le vecteur $\vec{v}$ tourné d'un angle $\alpha$ autour de l'axe dirigé
par $\vec{u}$.La rotation se fait dans le sens des aiguilles d'une montre 
si notre ligne de vue pointe dans la même direction que $\vec{u}$.Cette opération 
est connue comme la conjugaison par $q$.

Il s'ensuit que la multiplication de quaternions correspond à la composition de 
rotations, car si $p \text{et} q$ sont des quaternions représentant des rotations, 
alors la rotation (conjugaison) par $pq$ est:
\[
pq\vec{v}( pq )^-1 = pq\vec{v}q^-1p^-1 = p( q\vec{v}q^-1 )p^-1 \text{,}
\]
ce qui revient à tourner (conjuguer) par $q$, puis par $p$.

Le quaternion inverse d'une rotation correspond à la rotation inverse, car
$q^-1( q\vec{v}q^-1 )q = \vec{v}$.

Le carré d'un quaternion correspond à la 
rotation de deux fois le même angle autour du même axe. Plus généralement, 
$q^n$ correspond à une rotation de $n$ fois l'angle autour du même axe que $q$.
Cela peut être étendu à un réel arbitraire n, ce qui permet de calculer des rotations intermédiaires de façon fluide entre des rotations de l'espace, c'est
\emph{l'interpolation linéaire sphérique}.
		\subsubsection{Démonstration de l'équivalence entre conjugaison de quaternions et rotation de l'espace}
			Soit $\vec{u}$ un vecteur unitaire (l'axe de rotation) et soit 
$q = \cos \frac{\alpha}{2} + \vec{u} \sin \frac{\alpha}{2}$.

Notre but est de montrer que :

\[
	\vv{v'} = q\vec{v}q^-1 = ( \cos \frac{\alpha}{2} + 
	\vec{u} \sin \frac{\alpha}{2} ) \vv{v} ( \cos 
	\frac{\alpha}{2} - \vv{u} \sin \frac{\alpha}{2} )
\]

renvoie le vecteur $\vec{v}$ tourné d'un angle $\alpha$ autour de l'axe
dirigé par $\vec{u}$.

En développant, on obtient en effet :

\begin{align*}
	\vv{v'} & = \vv{v} \coscad + (\vec{u} \vec{v} - \vec{v}\vec{u}) \sinad \cosad- \vec{u}\vec{v}\vec{u}\sincad \\
	& = \vv{v}\coscad + 2 ( \vecsc{u}{v}) \sinad \cosad - (\vec{v}(\vec{u} 
	\cdot \vec{u})) \sincad \\
	& = \vec{v}( \coscad - \sincad ) + (\vecsc{u}{v}) (2 \sinad \cosad) + \vv{u}(\vecpr{u}{v}) (2
	\sincad) \\ 
	& = \vec{v} \cosa + (\vecsc{u}{v}) \sina + \vec{u}(\vecpr{u}{v})( 1 - \cosa ) \\
	& = (\vec{v} - \vec{u}(\vecpr{u}{v})) \cosa + (\vecsc{u}{v}) \sina + \vec{u}(\vecpr{u}{v}) \\
	& = \vec{v}_{\perp} \cosa + (\vecsc{u}{v}_{\perp}) \sina + \vec{v}_{\|}
\end{align*}

où $\vec{v}_{\perp} \text{ et } \vec{v}_{\|}$ sont les composantes de $\vec{v}$ 
respectivement orthogonale et colinéaire à $\vec{u}$. C'est là la formule de 
\bsc{Olinde Rodrigues} qui donne la rotation d'angle $\alpha$ autour de l'axe dirigé par $\vec{u}$.
	\subsection{Exemple}
		\begin{figure}[ht]
	\centering
	\includegraphics[height=5cm]{ressources/rotation_diagonale}\hfill
	\caption{Une rotation de \ang{120} autour de la première diagonale permute i, j et k circulairement.}
	\label{exemple}
\end{figure}

Considérons la rotation $f$ autour de l'axe dirigé par $ \vec{u} = i + j + k $ et 
d'angle \ang{120}, c'est-à-dire $\frac{2 \pi}{3}$ radians.

\[
\alpha = \frac{2 \pi}{3}
\]

La norme de $\vec{u}$ est $\sqrt{3}$, le demi-angle est $\frac{\pi}{3}$ (\ang{60}), le cosinus
de ce demi-angle est $\frac{1}{2}$, (cos \ang{60} = 0.5) et son sinus est $\sqrt{\frac{3}{2}}$,
(sin \ang{60} $\approx$ 0,866). Nous devons donc conjuguer avec le quaternion unitaire:

\begin{align*}
  q & = \cosad + \sinad \cdot \frac{1}{\lVert\vec{u}\rVert} \vec{u} \\
 & = \cos \frac{\pi}{3} + \sin \frac{\pi}{3} \cdot \frac{1}{\sqrt{3}} \vec{u} \\
 & = \frac{1}{2} + \frac{\sqrt{3}}{2} \cdot \frac{1}{\sqrt{3}} \vec{u} \\
 & = \frac{1}{2} + \frac{\sqrt{3}}{2} \cdot \frac{i + j + k}{\sqrt{3}} \\
 & = \frac{1 + i + j + k}{2}
\end{align*}

Si $f$ est la fonction de rotation,

\[
	f(ai+bj+ck) = q(ai+bj+ck)q^-1^-1
\]

On peut prouver que l'on obtient l'inverse d'un quaternion unitaire simplement en
changeant le signe de ses coordonnées imaginaires. En conséquence, 

\[
	q^-1 = \frac{1 - i - j - k}{2}
\]

et

\[
	f(ai+bj+ck) = \frac{1 - i - j - k}{2} (ai+bj+ck)\frac{1 - i - j - k}{2}
\]

En appliquant les règles ordinaires de calcul avec les quaternions, on obtient

\[
	f(ai+bj+ck) = ci + aj + ck
\]

Comme on s'y attendait, la rotation revient à tenir un cube par un de ses sommets, 
puis à le faire tourner de \ang{120} selon la diagonale la plus longue qui passe par ce point. On observe comment les trois axes subissent une \emph{permutation circulaire}.

\begin{dem}[title=Démonstration]{}{}
Prouvons le résultat précédent : 
En développant l'expression de $f$ (en deux étapes) et en appliquant les règles :

\[
	ij = k, ji = -k,
	jk=i, kj = -i,
	ki=k, ik=-j,
	i^2=j^2=k^2=-1
\]

on obtient:

\begin{align*}
	f(ai+bj+ck) &= \frac{1+i+j+k}{2}(ai+bj+ck)\frac{1-i-j-k}{2}
	(1 - i - j - k) \\
	&= \qter((ai+bj+ck)+(-a+bk-cj)+(-ak-b+ci)+(aj-bi-c)) \\
	(1 - i - j - k) \\
	&= \qter ((-a-b-c) + (a-b+c)i + (a+b-c)j + (-a+b+c)k) \\
	(1 - i - j - k)\\
	\text{...}
	&= ci + aj + bk
\end{align*}

Ce qui est bien le résultat annoncé. On voit que de tels calculs sont 
relativement fastidieux à faire à la main, mais dans un programme d'ordinateur, 
cela se résume à appeler deux fois la routine de multiplication de quaternions.

\end{dem}

	
\section{Expliquer les propriétés des quaternions à l'aide des rotations}
	\subsection{Non-commutativité}
		La multiplication des quaternions est non commutative. Comme multiplier des quaternions
unitaires revient à composer les rotations dans l'espace à trois dimensions, 
on peut rendre cette propriété intuitive grâce au fait que les rotations en trois dimensions ne commutent pas en général.

Un simple exercice consistant à appliquer deux rotations successives à un objet
asymétrique (par exemple un livre) peut l'expliquer. D'abord, tournez un livre 
de \ang{90} dans le sens des aiguilles d'une montre autour de l'axe des Z. 
Ensuite, basculez-le de \ang{180} autour de l'axe des X et mémorisez le résultat. Revenez à la position de départ, de manière à pouvoir lire à nouveau le titre du livre,
et appliquez les rotations en ordre inverse. 
Comparez le résultat au résultat précédent. Cela montre que, en général, 
la composition de deux rotations différentes autour de deux axes distincts de l'espace 
ne commute pas.
	\subsection{Les quaternions sont-ils orientés ?}
		Il convient de remarquer que les quaternions, comme n'importe quelle rotation ou application linéaire, ne sont pas "orientés" (il n'y a pas de sens direct ou indirect).
L'orientation d'un système de coordonnées provient de l'interprétation des nombres 
dans l'espace physique. Quelle que soit la convention d'orientation que l'on choisisse,
faire tourner le vecteur X de \ang{90} autour du vecteur Z renverra le vecteur Y — 
la théorie et les calculs donnent le même résultat.
	
\section{Les quaternions et les autres représentations des rotations}
	\subsection{Description qualitative des avantages des quaternions}
		La représentation d'une rotation sous la forme d'un quaternion (4 nombres) est plus compacte que la représentation en tant que matrice orthogonale (9 nombres). De plus, pour un axe et un angle donné, on peut facilement construire le quaternion correspondant, et réciproquement, pour un quaternion donné, on peut facilement extraire l'axe et l'angle. Toutes ces opérations sont beaucoup plus difficiles avec des matrices ou des angles d'Euler.

Dans les jeux vidéo et dans d'autres applications, on a souvent besoin de \og rotations fluides \fg{}, c'est-à-dire que la scène représentée doit tourner harmonieusement et pas d'un seul coup. On peut obtenir ce résultat en choisissant une courbe comme celle de l'interpolation linéaire sphérique dans l'espace des quaternions, avec une extrémité qui est la transformation identique 1 (ou correspondant à une autre rotation initiale) et l'autre extrémité correspondant à la rotation finale désirée. C'est plus difficile à faire avec d'autres représentations des rotations.

Quand on compose plusieurs rotations sur un ordinateur, les erreurs d'arrondi s'accumulent forcément. Un quaternion qui est légèrement erroné représente toujours une rotation après avoir été renormalisé ; une matrice qui est légèrement erronée ne sera plus orthogonale et sera difficile à convertir à nouveau en une matrice orthogonale qui convienne.

Les quaternions évitent également un phénomène appelé le blocage de cardan qui peut apparaître lorsque, par exemple dans des systèmes de rotations décrivant le roulis, le tangage et le lacet, le tangage est de \ang{90} vers le haut ou le bas, de telle façon que le roulis et le lacet correspondent au même mouvement, et qu'un degré de liberté ait été perdu. Dans un système de navigation inertielle à base de cardans, par exemple, cela peut avoir des conséquences désastreuses si l'avion monte en flèche ou descend à pic.
	\subsection{Conversion vers et depuis la représentation sous forme de matrice}
		\subsubsection{D'un quaternion en matrice orthogonale}
			La matrice orthogonale correspondant à une rotation au moyen du quaternion 
unitaire $z = a + bi + cj + dk (\text{avec} |z| = 1)$ est donnée par:

\[
\left(
\begin{tabular}{c c c}
$a^2+b^2-c^2-d^2$ & $2bc-2ad$ & $2ac+2bd$ \\
$2ad + 2bc$ & $a^2-b^2+c^2-d^2$ & $2cd-2ab$ \\
$2bd-2ac$ & $2ab+2cd$ & $a^2-b^2-c^2+d^2$ \\
\end{tabular}
\right)
\]
		\subsubsection{D'une matrice orthogonale en quaternion}
			Chercher le quaternion $(q_{0} + q_{x}i+q_{y}j + q_{z}k)$ correspondant à 
la matrice de rotation $Q_{ij}$ peut être instable numériquement si la trace
(la somme des éléments de la diagonale de la matrice) de la matrice de rotation 
est nulle ou très petite.
Une méthode robuste consiste à choisir l'élément de la diagonale ayant la valeur
$Q_{uu}$ la plus grande en valeur absolue. Considérons:

\[
r = \pm \frac{1}{2} \sqrt{1 + Q_{uu} + Q_{vv} + Q_{ww}}
\]

Cette écriture est légitime car l'expression sous la racine est positive. 
Si $r$ est nul, cela équivaut à une trace égale à $ - 1$ (soit un angle de rotation de $\pi$). La rotation est une symétrie axiale et donc le quaternion a la 
forme suivante $(0, \vec{u} x, \vec{u} y, \vec{u} z)$ avec $\vec{u}$ vecteur
unitaire dirigeant l'axe de rotation. Sinon le quaternion peut alors être 
écrit sous la forme:

\begin{align*}
	q_{0} &= r \\
	q_{u} &= \frac{1}{4r}(Q_{wv} - Q_{vw}) \\
	q_{v} &= \frac{1}{4r}(Q_{uv} - Q_{wu}) \\
	q_{w} &= \frac{1}{4r}(Q_{vu} - Q_{uv})
\end{align*}

Attention, il y a deux conventions pour les vecteurs : l'une suppose que les matrices de rotation 
sont multipliées par des vecteurs-ligne à gauche, l'autre par des vecteurs-colonne à droite; 
les deux conventions conduisent à des matrices qui sont transposées l'une de l'autre. 
La matrice ci-dessus suppose que l'on utilise des vecteurs colonne à droite. Historiquement, 
la convention vecteur-colonne-à-droite provient des mathématiques et de la mécanique classique, 
alors que la convention vecteur-ligne-à-gauche provient de l'infographie, où il était plus 
facile de saisir des vecteurs-lignes aux débuts de la discipline.
		\subsubsection{Quaternions optimaux}
			La section ci-dessus a décrit comment récupérer un quaternion $q$ à partir d'une matrice de rotation 
$3\times3 \text{ } Q$. Supposons, néanmoins, que nous avons une matrice $Q$ qui n'est pas celle d'une rotation pure, 
à cause d'erreurs d'arrondi, par exemple, et que nous souhaitions trouver le quaternion $q$ qui représente 
$Q$ le plus précisément possible. Dans ce cas, nous construisons une matrice $4 \times 4$ symétrique:

\[
	K = \frac{1}{3}
	\left[
		\begin{tabular}{c c c c}
			$\Qxx - \Qyy -\Qzz$ & $\Qyx + \Qxy$ & $\Qzx + \Qxz$ & $\Qyz - \Qzy$ \\
			$\Qyx + \Qxy$ & $\Qyy - \Qxx - \Qzz$ & $\Qzy + \Qyz$ & $\Qzx - \Qxz$ \\
			$\Qzx + \Qxz$ & $\Qzy + \Qyz$ & $\Qzz - \Qxx - \Qyy$ & $\Qxy - \Qyx$ \\
			$\Qyz - \Qzy$ & $\Qzx - \Qxz$ & $\Qxy - \Qyx$ & $\Qxx + \Qyy \Qzz$
		\end{tabular}
	\right]
\]

et nous cherchons le vecteur propre de coordonnées $(x, y, z, w)$ correspondant 
à la plus grande valeur propre (cette valeur vaut 1 si et seulement si $Q$ 
est une rotation pure). Le quaternion ainsi obtenu correspondra à la rotation 
la plus proche de la matrice $Q$ de départ.
	\subsection{Comparaisons de performances avec d'autres méthodes de rotation}
		Cette section traite des implications en termes de performances de l'utilisation de
quaternions par rapport à d'autres méthodes (axe et angle ou matrices de rotation) 
pour effectuer des rotations en 3D.

% Force les notes de page à se mettre au bon endroit lol
% Trick everywhere! 
\clearpage 
		\subsubsection{Résultats}
			\begin{table}[ht]
	\centering
		\begin{tabular}{cc}
			Méthode & Mémoire \\
			\toprule
			Matrice de rotation & 9 \\
			\midrule
			Quaternion & 4 \\
			\midrule
			Axe et angle & 4* \\
			\bottomrule
		\end{tabular}
	\caption{Utilisation en mémoire}
	\label{memoire}
\end{table}

\begin{table}[ht]
	\centering
	\begin{tabular}{cccc}
		Méthode & Multiplications & Additions et soustractions & Nombre total d'opérations \\
		\toprule
		Matrice de rotation & 27 & 18 & 45 \\
		\midrule
		Quaternion & 16 & 12 & 28 \\
		\bottomrule
	\end{tabular}
	\caption{Comparaison de performances de la composition de rotations}
	\label{performances_rotations}
\end{table}

\begin{table}[h]
	\centering
	\begin{tabular}{ccccc}
		Méthode & Multiplications & Additions et soustractions & sin et cos & Nombre total d'opérations \\
		\toprule
		Matrice de rotation & 9 & 6 & 0 & 15 \\
		\midrule
		Quaternion & 21 & 18 & 0 & 39 \\
		\midrule
		Axe et angle & 23 & 16 & 2 & 41 \\
		\bottomrule
	\end{tabular}
	\caption{Comparaison de performances de la composition de vecteurs}
	\label{performances_vecteurs}
\end{table}



\footnote{* Note : la représentation sous forme d'angle et d'axe peut
être stockée dans 3 emplacements seulement en multipliant l'axe de rotation 
par l'angle de rotation ; néanmoins, avant de l'utiliser, il faut récupérer 
le vecteur unitaire et l'angle en re-normalisant, ce qui coûte des opérations mathématiques supplémentaires.}
		\subsubsection{Méthodes utilisées}
			Il y a trois techniques de base pour faire tourner un vecteur $\vec{v}$ :
\begin{enumerate}
	\item Calculer le produit matriciel de la matrice de rotation $3\times3$ représentant la 
	rotation \textbf{R} par la matrice colonne $3\times1$ représentant le vecteur $\vec{v}$.
	Cela nécessite $3 \times (3 \text{multiplications} + 2 \text{additions}) = $ 
	9 multiplications et 6 additions, c'est la méthode la plus efficace pour faire tourner 
	un vecteur.
	
	\item Utiliser la formule de rotation avec des quaternions dite \emph{action par conjugaison} $\vec{v_{\text{new}}} = z\vec{v}z^-1$.  Calculer ce résultat 
	revient à transformer le quaternion en \emph{une matrice de rotation} 
	\textbf{R}  en utilisant la formule de conversion d'un \emph{quaternion 
	en matrice orthogonale}, puis à multiplier le résultat par la
	matrice-colonne représentant le vecteur. 
	En procédant à une \emph{recherche de sous-expressions communes}, l'algorithme 
	se ramène à 21 multiplications et 18 additions. Une autre approche peut 
	consister à convertir d'abord le quaternion dans la représentation sous 
	forme d'axe et d'angle équivalente, puis à s'en servir pour faire tourner 
	le vecteur. Néanmoins, c'est à la fois moins efficace et moins stable 
	numériquement lorsque le quaternion est voisin de la rotation identité.
	
	\item Convertir la représentation sous forme d'axe et d'angle en 
	\emph{matrice de rotation} \textbf{R}, puis multiplier par la matrice-colonne 
	représentant le vecteur. La conversion coûte 14 multiplications, 
	2 appels de fonction (sin et cos) et 10 additions ou soustractions 
	après \emph{recherche de sous-expressions communes}, puis la rotation ajoute 
	9 multiplications et 6 additions pour un total de 23 multiplications, 
	16 additions ou soustractions et deux appels de fonctions trigonométriques.
\end{enumerate}
\section{Les paires de quaternions unitaires comme rotations dans l'espace à 4 dimensions}