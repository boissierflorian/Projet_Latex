\DeclarePairedDelimiter\abs{\lvert}{\rvert}% % Valeur absolue

\tcbset{fonttitle=\bfseries,separator sign none, description delimiters parenthesis}
\newtcbtheorem{dem}{Démonstration}{colback=green!5!white,colframe=green!75!black}{}

\makesavenoteenv{tabular}

% Vecteur 3d affiché en colonne
% Param #1: le nom du vecteur
% Param #2-4: x, y et z
\newcommand{\vvv}[4]{%
	\vv{#1}\left(
		\begin{tabular}{c}
			#2 \\ #3 \\ #4 \\
		\end{tabular}
	\right)
}

% cos alpha/2
\newcommand{\cosad}{	
	\cos \frac{\alpha}{2}
}

% sin alpha/2
\newcommand{\sinad}{	
	\sin \frac{\alpha}{2}
}

% cos² alpha/2
\newcommand{\coscad}{	
	\cos^2 \frac{\alpha}{2}
}

% sin² alpha/2
\newcommand{\sincad}{	
	\sin^2 \frac{\alpha}{2}
}

% cos alpha
\newcommand{\cosa}{\cos \alpha}

% sin alpha
\newcommand{\sina}{\sin \alpha}

% v ^ u
\newcommand{\vecsc}[2]{
	\vec{#1} \wedge \vec{#2}
}

% v . u
\newcommand{\vecpr}[2]{
	\vec{#1} \cdot \vec{#2}
}

% 1 / 4
\newcommand{\qter}{
	\frac{1}{4}
}

% Ecriture simplifiées
\newcommand{\Q}[2]{
	Q_{#1#2}
}

\newcommand{\Qxx}{
	\Q{x}{x}
}

\newcommand{\Qxy}{
	\Q{x}{y}
}

\newcommand{\Qxz}{
	\Q{x}{z}
}

\newcommand{\Qyx}{
	\Q{y}{x}
}

\newcommand{\Qyy}{
	\Q{y}{y}
}

\newcommand{\Qyz}{
	\Q{y}{z}
}

\newcommand{\Qzx}{
	\Q{z}{x}
}

\newcommand{\Qzy}{
	\Q{z}{y}
}

\newcommand{\Qzz}{
	\Q{z}{z}
}

% Note d'un élément d'un tableau (cf. boissier/subsections/resultats.tex)
\newcommand{\fnoteone}{
	\footnote{la représentation sous forme d'angle 
	et d'axe peut être stockée dans 3 emplacements seulement 
	en multipliant l'axe de rotation 
	par l'angle de rotation ; néanmoins, avant de l'utiliser, 
	il faut récupérer le vecteur unitaire et l'angle en 
	re-normalisant, ce qui coûte des opérations mathématiques supplémentaires.}
}

% Matrice 1 colonne et 4 lignes
\newcommand{\moq}[4]{
	\left(
		\begin{tabular}{c}
			$#1$ \\
			$#2$ \\
			$#3$ \\
			$#4$
		\end{tabular}
	\right)
}