On peut définir les nombres complexes en introduisant un symbole abstrait $i$ qui se conforme aux 
règles usuelles de l'algèbre et qui en plus obéit à la règle $i^2 = - 1$. 
Cela suffit à reproduire toutes les règles de calcul des nombres complexes, 
par exemple: 
\[
(a + bi)(c + di) = ac + adi + bic + bidi = ac + adi + bci + bdi^{2} = (ac - bd) + (bc + ad) i
\]

De la même façon, les quaternions peuvent être définis en introduisant des symboles abstraits 
$i, j \text{et} k$ qui satisfont aux règles $i^2 = j^2 = k^2 = ijk = -1$ et les règles 
algébriques usuelles \emph{sauf} la commutativité de la multiplication (un exemple familier 
de multiplication non commutative est la multiplication des matrices). L'ensemble des règles 
de calcul découle de ces définitions ; par exemple, on peut montrer que 
\begin{align*}
(a+bi+cj+dk)(e+fi+gj+hk) &= (ae-bf-cg-dh)+(af+be+ch-dg)i+ \\
(ag+ce+df-bh)j + (ah+de+bg-cf)k.
\end{align*}

La partie imaginaire $bi+cj+dk$ d'un quaternion se comporte comme un vecteur $\vvv{v}{b}{c}{d}$
 d'un espace vectoriel à trois dimensions et la partie réelle $a$ comme un scalaire de $\mathbb{R}$.

Quand les quaternions sont utilisés en géométrie, il est pratique de les définir comme un 
scalaire plus un vecteur: $a+bi+cj+dk = a + \vec{v}$

Ceux qui ont étudié les vecteurs à un niveau élémentaire pourraient trouver étrange d'additionner 
un \emph{nombre} à un vecteur, car ce sont des objets de natures très différentes, ou de 
\emph{multiplier} deux vecteurs entre eux, car cette opération n'est d'habitude pas définie.
 Néanmoins, si l'on se souvient qu'il ne s'agit là que d'une notation pour les parties réelles et 
 imaginaires d'un quaternion, cela devient plus légitime.

Nous pouvons exprimer la multiplication de quaternions (produit de Hamilton) dans le langage moderne 
du produit vectoriel et du produit scalaire de vecteurs (qui ont en fait été inspirés au début par 
les quaternions). À la place des règles $i^2=j^2=k^2=ijk=-1$, nous avons la règle de multiplication 
de deux vecteurs $\vec{v}\vec{w}=\vec{v}\wedge\vec{w}-\vec{v}\cdot\vec{w}$, où:

\begin{itemize}
	\item $\vec{v}\vec{w}$ est la multiplication de vecteurs,
	\item $\vec{v}\wedge\vec{w}$ est le produit vectoriel (un vecteur),
	\item $\vec{v}\cdot\vec{w}$ est le produit scalaire un nombre).s
\end{itemize}

La multiplication de vecteurs n'est pas commutative (à cause du produit vectoriel), alors que la 
multiplication entre scalaires et entre un scalaire et un vecteur sont commutatives. Il découle de
 manière immédiate de ces règles que $(s + \vec{v})(t + \vec{w}) = (st - \vec{v}\cdot\vec{w}) + 
 (s\vec{w}+t\vec{v}+\vec{v}\wedge\vec{w})$.

L'inverse (à gauche et à droite) d'un quaternion non nul est $(s + \vec{v})^-1 = 
\frac{s - \vec{v}}{s^2 + \abs{\vec{v}}^2}$, comme cela peut être vérifié par calcul direct.