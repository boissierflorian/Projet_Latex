\begin{table}[ht]
	\centering
		\begin{tabular}{cc}
			Méthode & Mémoire \\
			\toprule
			Matrice de rotation & 9 \\
			\midrule
			Quaternion & 4 \\
			\midrule
			Axe et angle & 4* \\
			\bottomrule
		\end{tabular}
	\caption{Utilisation en mémoire}
	\label{memoire}
\end{table}

\begin{table}[ht]
	\centering
	\begin{tabular}{cccc}
		Méthode & Multiplications & Additions et soustractions & Nombre total d'opérations \\
		\toprule
		Matrice de rotation & 27 & 18 & 45 \\
		\midrule
		Quaternion & 16 & 12 & 28 \\
		\bottomrule
	\end{tabular}
	\caption{Comparaison de performances de la composition de rotations}
	\label{performances_rotations}
\end{table}

\begin{table}[h]
	\centering
	\begin{tabular}{ccccc}
		Méthode & Multiplications & Additions et soustractions & sin et cos & Nombre total d'opérations \\
		\toprule
		Matrice de rotation & 9 & 6 & 0 & 15 \\
		\midrule
		Quaternion & 21 & 18 & 0 & 39 \\
		\midrule
		Axe et angle & 23 & 16 & 2 & 41 \\
		\bottomrule
	\end{tabular}
	\caption{Comparaison de performances de la composition de vecteurs}
	\label{performances_vecteurs}
\end{table}



\footnote{* Note : la représentation sous forme d'angle et d'axe peut
être stockée dans 3 emplacements seulement en multipliant l'axe de rotation 
par l'angle de rotation ; néanmoins, avant de l'utiliser, il faut récupérer 
le vecteur unitaire et l'angle en re-normalisant, ce qui coûte des opérations mathématiques supplémentaires.}