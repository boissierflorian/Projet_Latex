La multiplication des quaternions est non commutative. Comme multiplier des quaternions
unitaires revient à composer les rotations dans l'espace à trois dimensions, 
on peut rendre cette propriété intuitive grâce au fait que les rotations en trois dimensions ne commutent pas en général.

Un simple exercice consistant à appliquer deux rotations successives à un objet
asymétrique (par exemple un livre) peut l'expliquer. D'abord, tournez un livre 
de \ang{90} dans le sens des aiguilles d'une montre autour de l'axe des Z. 
Ensuite, basculez-le de \ang{180} autour de l'axe des X et mémorisez le résultat. Revenez à la position de départ, de manière à pouvoir lire à nouveau le titre du livre,
et appliquez les rotations en ordre inverse. 
Comparez le résultat au résultat précédent. Cela montre que, en général, 
la composition de deux rotations différentes autour de deux axes distincts de l'espace 
ne commute pas.