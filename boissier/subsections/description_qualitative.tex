La représentation d'une rotation sous la forme d'un quaternion (4 nombres) est plus compacte que la représentation en tant que matrice orthogonale (9 nombres). De plus, pour un axe et un angle donné, on peut facilement construire le quaternion correspondant, et réciproquement, pour un quaternion donné, on peut facilement extraire l'axe et l'angle. Toutes ces opérations sont beaucoup plus difficiles avec des matrices ou des angles d'Euler.

Dans les jeux vidéo et dans d'autres applications, on a souvent besoin de \og rotations fluides \fg{}, c'est-à-dire que la scène représentée doit tourner harmonieusement et pas d'un seul coup. On peut obtenir ce résultat en choisissant une courbe comme celle de l'interpolation linéaire sphérique dans l'espace des quaternions, avec une extrémité qui est la transformation identique 1 (ou correspondant à une autre rotation initiale) et l'autre extrémité correspondant à la rotation finale désirée. C'est plus difficile à faire avec d'autres représentations des rotations.

Quand on compose plusieurs rotations sur un ordinateur, les erreurs d'arrondi s'accumulent forcément. Un quaternion qui est légèrement erroné représente toujours une rotation après avoir été renormalisé ; une matrice qui est légèrement erronée ne sera plus orthogonale et sera difficile à convertir à nouveau en une matrice orthogonale qui convienne.

Les quaternions évitent également un phénomène appelé le blocage de cardan qui peut apparaître lorsque, par exemple dans des systèmes de rotations décrivant le roulis, le tangage et le lacet, le tangage est de \ang{90} vers le haut ou le bas, de telle façon que le roulis et le lacet correspondent au même mouvement, et qu'un degré de liberté ait été perdu. Dans un système de navigation inertielle à base de cardans, par exemple, cela peut avoir des conséquences désastreuses si l'avion monte en flèche ou descend à pic.