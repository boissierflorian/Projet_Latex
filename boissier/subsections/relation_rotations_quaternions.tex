Soient $(w, x, y, z)$ les coordonnées d'une rotation, comme décrit précédemment. 
Définissons le quaternion : $q = w + xi + yj+ zk = w + \vvv{v}{x}{y}{z} = 
\cos( \alpha / 2 ) + \vec{u} \sin( \alpha/2 )$
où $\vec{u}$ est un vecteur unitaire. Soit également $\vec{v}$ un vecteur
ordinaire de l'espace en 3 dimensions, considéré comme un quaternion avec
une coordonnée réelle nulle. On peut alors montrer (voir section suivante) 
que le produit de quaternions: 
\[
q\vec{v}q^-1
\]
renvoie le vecteur $\vec{v}$ tourné d'un angle $\alpha$ autour de l'axe dirigé
par $\vec{u}$.La rotation se fait dans le sens des aiguilles d'une montre 
si notre ligne de vue pointe dans la même direction que $\vec{u}$.Cette opération 
est connue comme la conjugaison par $q$.

Il s'ensuit que la multiplication de quaternions correspond à la composition de 
rotations, car si $p \text{et} q$ sont des quaternions représentant des rotations, 
alors la rotation (conjugaison) par $pq$ est:
\[
pq\vec{v}( pq )^-1 = pq\vec{v}q^-1p^-1 = p( q\vec{v}q^-1 )p^-1 \text{,}
\]
ce qui revient à tourner (conjuguer) par $q$, puis par $p$.

Le quaternion inverse d'une rotation correspond à la rotation inverse, car
$q^-1( q\vec{v}q^-1 )q = \vec{v}$.

Le carré d'un quaternion correspond à la 
rotation de deux fois le même angle autour du même axe. Plus généralement, 
$q^n$ correspond à une rotation de $n$ fois l'angle autour du même axe que $q$.
Cela peut être étendu à un réel arbitraire n, ce qui permet de calculer des rotations intermédiaires de façon fluide entre des rotations de l'espace, c'est
\emph{l'interpolation linéaire sphérique}.