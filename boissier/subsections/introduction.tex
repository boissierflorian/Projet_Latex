Les quaternions unitaires fournissent une notation mathématique commode pour 
représenter l'orientation et la rotation d'objets en trois dimensions. 
Comparés aux angles d'Euler, ils sont plus simple à composer et évitent le problème 
du blocage de cardan. Comparés aux matrices de rotations, ils sont plus stables numériquement et peuvent se révéler plus efficaces. Les quaternions ont été adoptés 
dans des applications en infographie, robotique, navigation, dynamique moléculaire 
et la mécanique spatiale des satellites.