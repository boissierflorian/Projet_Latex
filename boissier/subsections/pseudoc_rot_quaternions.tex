Étant donné un quaternion $z = a + bi + cj + dk (avec \abs{z} = 1)$ et un vecteur
$\vec{v}$ de coordonnées \emph{v1, v2} et \emph{v3}, le code qui suit 
effectue une rotation. Notez l'utilisation de variables temporaires 
$txx$. Notez aussi l'optimisation des éléments de la diagonale de la matrice
\textbf{R}: comme $a^2 + b^2 + c^2 + d^2 = 1$, on réécrit l'élément en 
haut à gauche sous la forme $a^2 + b^2 + c^2 + d^2 -2c^2 - 2d^2 = 1 - 2c^2 - 2d^2$;
les deux autres éléments de la diagonale peuvent être réécrits de même.

\lstset{frame=single,framesep=2pt,
	aboveskip=1ex,numberstyle=\tiny\color{blue},
}

\lstinputlisting{boissier/listings/pseudocode_quat.txt}

\clearpage