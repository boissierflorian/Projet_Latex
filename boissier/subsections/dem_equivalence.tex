Soit $\vec{u}$ un vecteur unitaire (l'axe de rotation) et soit 
$q = \cos \frac{\alpha}{2} + \vec{u} \sin \frac{\alpha}{2}$.

Notre but est de montrer que :

\[
	\vv{v'} = q\vec{v}q^-1 = ( \cos \frac{\alpha}{2} + 
	\vec{u} \sin \frac{\alpha}{2} ) \vv{v} ( \cos 
	\frac{\alpha}{2} - \vv{u} \sin \frac{\alpha}{2} )
\]

renvoie le vecteur $\vec{v}$ tourné d'un angle $\alpha$ autour de l'axe
dirigé par $\vec{u}$.

En développant, on obtient en effet :

\begin{align*}
	\vv{v'} & = \vv{v} \coscad + (\vec{u} \vec{v} - \vec{v}\vec{u}) \sinad \cosad- \vec{u}\vec{v}\vec{u}\sincad \\
	& = \vv{v}\coscad + 2 ( \vecsc{u}{v}) \sinad \cosad - (\vec{v}(\vec{u} 
	\cdot \vec{u})) \sincad \\
	& = \vec{v}( \coscad - \sincad ) + (\vecsc{u}{v}) (2 \sinad \cosad) + \vv{u}(\vecpr{u}{v}) (2
	\sincad) \\ 
	& = \vec{v} \cosa + (\vecsc{u}{v}) \sina + \vec{u}(\vecpr{u}{v})( 1 - \cosa ) \\
	& = (\vec{v} - \vec{u}(\vecpr{u}{v})) \cosa + (\vecsc{u}{v}) \sina + \vec{u}(\vecpr{u}{v}) \\
	& = \vec{v}_{\perp} \cosa + (\vecsc{u}{v}_{\perp}) \sina + \vec{v}_{\|}
\end{align*}

où $\vec{v}_{\perp} \text{ et } \vec{v}_{\|}$ sont les composantes de $\vec{v}$ 
respectivement orthogonale et colinéaire à $\vec{u}$. C'est là la formule de 
\bsc{Olinde Rodrigues} qui donne la rotation d'angle $\alpha$ autour de l'axe dirigé par $\vec{u}$.