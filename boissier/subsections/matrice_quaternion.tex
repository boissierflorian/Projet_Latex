Chercher le quaternion $(q_{0} + q_{x}i+q_{y}j + q_{z}k)$ correspondant à 
la matrice de rotation $Q_{ij}$ peut être instable numériquement si la trace
(la somme des éléments de la diagonale de la matrice) de la matrice de rotation 
est nulle ou très petite.
Une méthode robuste consiste à choisir l'élément de la diagonale ayant la valeur
$Q_{uu}$ la plus grande en valeur absolue. Considérons:

\[
r = \pm \frac{1}{2} \sqrt{1 + Q_{uu} + Q_{vv} + Q_{ww}}
\]

Cette écriture est légitime car l'expression sous la racine est positive. 
Si $r$ est nul, cela équivaut à une trace égale à $ - 1$ (soit un angle de rotation de $\pi$). La rotation est une symétrie axiale et donc le quaternion a la 
forme suivante $(0, \vec{u} x, \vec{u} y, \vec{u} z)$ avec $\vec{u}$ vecteur
unitaire dirigeant l'axe de rotation. Sinon le quaternion peut alors être 
écrit sous la forme:

\begin{align*}
	q_{0} &= r \\
	q_{u} &= \frac{1}{4r}(Q_{wv} - Q_{vw}) \\
	q_{v} &= \frac{1}{4r}(Q_{uv} - Q_{wu}) \\
	q_{w} &= \frac{1}{4r}(Q_{vu} - Q_{uv})
\end{align*}

Attention, il y a deux conventions pour les vecteurs : l'une suppose que les matrices de rotation 
sont multipliées par des vecteurs-ligne à gauche, l'autre par des vecteurs-colonne à droite; 
les deux conventions conduisent à des matrices qui sont transposées l'une de l'autre. 
La matrice ci-dessus suppose que l'on utilise des vecteurs colonne à droite. Historiquement, 
la convention vecteur-colonne-à-droite provient des mathématiques et de la mécanique classique, 
alors que la convention vecteur-ligne-à-gauche provient de l'infographie, où il était plus 
facile de saisir des vecteurs-lignes aux débuts de la discipline.