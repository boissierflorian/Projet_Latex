%Il y a trois techniques de base pour faire tourner un vecteur $\vec{v}$ :
%\begin{enumerate}
%	\item Calculer le produit matriciel de la matrice de rotation $3\times3$ représentant la 
%	rotation \textbf{R} par la matrice colonne $3\times1$ représentant le vecteur $\vec{v}$.
%	Cela nécessite $3 \times (3 \text{multiplications} + 2 \text{additions) = $ 
%	9 multiplications et 6 additions, c'est la méthode la plus efficace pour faire tourner 
%	un vecteur.
%	\item Utiliser la formule de rotation avec des quaternions dite \emph{action par conjugaison}
%	 $\vec{v}$
%	\item L
%\end{enumerate}